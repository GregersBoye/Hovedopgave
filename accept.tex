\section{Accept test}
Sidst i sprint tre ville vi gerne have haft vores produkt owner og en s�lger til at lave en r�kke Accept test som vi har skrevet. Form�let med disse tests er at f� produkt owneren til at teste om produktet lever op til hans krav. Der er ofte langt fra beskrivelserne i user stories til det brugerne rent faktisk mente. Ved hele tiden at have vist de nye metoder i vores sprint reviews burde der ikke v�re s� mange overraskelser p� den front. Det er mere funktionelle fejl, og test om brugerne kan finde ud af brugergr�nsefladen. Ud fra den respons vi ville have f�et kunne vi have planlagt vores fjerde sprint med de rettelser og eventuelle fejl der ville blive fundet ville finde. Vi vil starte med at teste nogle af systemets vigtigste funktioner. 

Vi n�ede desv�rre aldrig at f� sat vores accept test i gang. Vi ville gerne have lavet en bruger til de to testere, og s� egentlig bare s�tte dem foran systemet og lade dem k�re vores accept tests igennem. De skal selvf�lgelig tage noter omkring hvordan de oplever at bruge systemet. Samt skrive de fejl ned de oplever, og eventuelle m�der tingene kan g�res smartere for dem. 

Vi mener det er vigtigt at dem der rent faktisk skal bruge systemet er med til at teste det.

\subsection{Filter Leads}
Denne metode bruges til at filtre leads i oversigten. P� den m�de kan man finde frem til pr�cis de leads man �nsker at skulle arbejde med.

\begin{itemize}
\item G� ind p� Leads oversigt siden
\item Find alle Leads med post nummer mellem 9000 og 9400
\item Derefter filtre dem fra 
\end{itemize}

\subsection{Upload CSV}
Her kan man uploade en r�kke leads i en CSV fil. P� den m�de kan man oprette mange leads afgangen frem for et.

\begin{itemize}
\item G� in p� Leads oversigt siden
\item Upload udleveret CSV fil til systemet
\item V�lg de Leads blandt CSV-dubletterne der er rigtige
\item Godkend de valgte leads
\item V�lg de Leads blandt DB-dubletterne der er rigtige
\end{itemize}

\subsection{Find Lead Details og Kontakt Details}
Her finder vi detaljer omkring et bestemt lead og dens kontakter. 

\begin{itemize}
\item Sorter firmaerne s� kun Leads bliver vist i menuen til venstre
\item Find stamdata omkring leadet ved at klikke p� navnet
\item Find stamdata omkring en Kontakt fra leadet ved at klikke p� navnet i h�jre side
\item Find og l�s en note ved at kigge p� note historie
\item Opret en note omkring firmaet ved at klikke p� note knappen
\item Ret i firmaets stamdata ved at klikke i feltet der skal rettes
\end{itemize}
