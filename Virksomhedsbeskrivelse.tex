\section{Virksomhedsbeskrivelse}
EqualSums er en relativt lille virksomhed der blev stiftet af Mikkel Elliott og Martin Skov Kristensen. Virksomheden besk�ftiger sig med udvikling af online softwarel�sninger til bogholderi og bogf�ring. Udover Mikkel og Martin er der en fast udviklerstab p� 4 backend-udviklere og �n frontendudvikler. Udover den faste stab er der pt. en konsulent, tre praktikanter og en studentermedarbejder. Udviklerstaben styres af en projektleder, der s�rger for den interne koordination i udviklergruppen.

Af �vrige ansatte er der tre s�lgere hvoraf en ogs� fungerer som administrativ medarbejder. 

Ledelsen fremmer bevidst en uh�jtidelig tone i virksomheden, hvilket bl.a. ses ved, at hele virksomheden spiser frokost sammen, og oftest g�r en kortere tur herefter. Medarbejderne har ogs� s�kkestole og boksebold til r�dighed, ligesom der ogs� er kaffe ad libitum. Herudover er man til enhver tid velkommen til at medbringe kage.

Der er mere eller mindre frie m�detider, s� l�nge man opn�r 37 timer ugentligt, dette giver medarbejderne stor frihed til selv at v�lge deres m�detider. 

Den oprindelige id� var, at virksomheden skulle tilbyde en hel vifte af softwarel�sninger. Da man for alvor kom i gang med den f�rste l�sning, blev det dog vurderet, at dette produkt i sig selv kunne holde hele virksomheden besk�ftiget. Derfor skiftede fokus til dette enkelte produkt, der idag er det EqualSums udvikler.

I salgsafdelingen bliver der i h�j grad brugt Googles services til at holde styr p� kontakter, kunder og dokumenter. Der bliver benyttet Google docs til dokumenter, kunder ligger i Google contacts, der automatisk synkroniseres med s�lgernes mobiltelefoner, og i et regneark bliver der holdt styr p� leads. Problemet er, at der ikke er noget der s� at sige, binder alle disse steder sammen. S�lgerne bruger en stor del af dagen p� at finde ud af hvem der skal kontaktes, opdatere oplysninger osv. De savner kort sagt et system der kan fungere som f�lles samlingspunkt for alle disse services.